\documentclass[12pt,a4paper]{article}
\usepackage{ctex} % 支持中文
\usepackage{geometry} % 设定页面边距
\usepackage{graphicx} % 插入图片
\usepackage{amsmath, amssymb} % 数学符号与公式
\usepackage{booktabs} % 美化表格
\usepackage{longtable} % 支持长表格
\usepackage{enumitem} % 调整列表格式
\usepackage{hyperref} % 超链接
\usepackage{titlesec} % 自定义标题格式
\usepackage{fancyhdr} % 页眉页脚
\usepackage{setspace} % 行距控制
\usepackage{caption} % 控制标题格式
\usepackage[numbers]{natbib} % 参考文献格式

% 设定页面边距
\geometry{top=2.54cm, bottom=2.54cm, left=3.17cm, right=3.17cm}

% 设定段落格式
\setlength{\parindent}{2em} % 首行缩进 2 个字符
\setlength{\parskip}{0em} % 段间距为 0

% 设定行距
\renewcommand{\baselinestretch}{1.5} % 1.5 倍行距

% 标题格式
\titleformat{\section}{\heiti\zihao{3}}{\thesection}{1em}{}
\titleformat{\subsection}{\kaishu\zihao{-4}}{\thesubsection}{1em}{}
\titleformat{\subsubsection}{\songti\zihao{-4}\bfseries}{\thesubsubsection}{1em}{}

% 页眉页脚
\pagestyle{fancy}
\fancyhf{}
\fancyhead[C]{\songti\zihao{-4} 2025 年(第十一届)全国大学生统计建模大赛}
\fancyfoot[C]{\thepage}

% 设置图表标题格式
\captionsetup{font={small, bf}, labelsep=space}

\begin{document}

% 封面
\begin{titlepage}
    \begin{center}
        {\heiti\zihao{2} 2025 年(第十一届)全国大学生统计建模大赛} \\
        \vspace{1cm}
        {\heiti\zihao{2} 参 赛 作 品} \\
        \vspace{2cm}
        {\heiti\zihao{3} 论文题目:XX 相对贫困治理成效的统计测度研究} \\
        \vspace{2cm}
        {\songti\zihao{4} 参赛学校:XXXXXX 大学} \\
        {\songti\zihao{4} 参赛队员:XXX、XXX、XXX} \\
        {\songti\zihao{4} 指导老师:XXX、XXX} \\
        \vfill
        {\songti\zihao{4} 作品编号:TJJM20250000000} \\
        \vspace{1cm}
    \end{center}
\end{titlepage}

% 摘要
\begin{center}
    {\heiti\zihao{3} 摘要}
\end{center}
\noindent
中国目前已彻底消除了绝对贫困,减贫事业将进入治理相对贫困的新阶段,但学界尚缺乏相应的统计指数来测度相对贫困的治理成效。为此,本文……
\vspace{1em}

\noindent
\textbf{关键词}:相对贫困;持续多维脱贫指数;……

\newpage
% 目录
\tableofcontents
\newpage

% 正文
\section{持续多维相对贫困脱/返贫指数的测算思路}
长期以来,世界各国都在为消除极端贫困而努力,并取得了一定的成绩。在全球范围内,极端贫困发生率迅速下降。据世界银行估计,极端贫困人口的比例从 1990 年的 36.2\% 下降到 2017 年的 9.3\%。极端贫困的问题……

\subsection{单维相对贫困的识别}
相对贫困发生率和相对贫困减贫成效都是一个整体的概念……个体福利用 $d$ 个指标来表示其水平指数,对某一区县,$x_{ij}^{(t)}$ 表示 $t$ 时期个体 $i$ 在福利指标 $j$ 上的取值……

\subsubsection{单维相对贫困脱贫成效/返贫现象的识别}
利用 Markov 链模型思路……其代表性元素 $p_{ij}^{(t)}$ 的计算过程如下:
\begin{equation}
    P^{(t)} =
    \begin{bmatrix}
        p_{00}^{(t)} & p_{01}^{(t)} \\
        p_{10}^{(t)} & p_{11}^{(t)}
    \end{bmatrix}
\end{equation}

\newpage
% 表格示例
\section{表格示例}
\begin{table}[htbp]
    \centering
    \caption{多维减贫成效测度的相对贫困维度指标}
    \begin{tabular}{llcc}
        \toprule
        指标层 & 指标解释与赋值 & 临界值 & 权重 \\
        \midrule
        人均纯收入 & 家庭人均年纯收入(2010 年不变价) & $0.4 \times$ 中位数 & 1/8 \\
        身体质量指数(BMI) & BMI 指数(体重 kg / 身高 m²) & $0.5 \times$ (中位数 $+18.5$) & 1/8 \\
        \bottomrule
    \end{tabular}
\end{table}

% 参考文献
\newpage
\begin{thebibliography}{99}
\bibitem{jiang2017} 蒋南平, 郑万军. 中国农民工多维返贫测度问题[J]. 中国农村经济, 2017(6):58-69.
\bibitem{wang2020} 汪三贵, 刘明月. 从绝对贫困到相对贫困: 理论关系、战略转变与政策重点[J]. 华南师范大学学报 (社会科学版), 2020(06):18-29+189.
\bibitem{airio2005} Airio I, Moisio P, Niemelä M. Intergenerational Transmission of Poverty in Finland in the 1990s[J]. European Journal of Social Security, 2005, 7(3):253-269.
\end{thebibliography}

% 附录
\newpage
\section*{附录}
\begin{table}[htbp]
    \centering
    \caption{简称与全称对应表}
    \begin{tabular}{ll}
        \toprule
        全称 & 简称 \\
        \midrule
        持续多维相对贫困脱贫指数 & $\alpha$ 相对脱贫指数 \\
        持续多维相对贫困脱贫成效 & $\alpha$ 相对脱贫成效 \\
        \bottomrule
    \end{tabular}
\end{table}

\section*{致谢}
论文的完成离不开 XXX 老师的指导……
\end{document}
